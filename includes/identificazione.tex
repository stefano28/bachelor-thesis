\section{Sistema di identificazione}
Il meccanismo di identificazione è la procedura per verificare che un determinato dispositivo
è abilitato a connettersi alla rete. Il 5G apporta una modifica organica del suo funzionamento, per
questo si è deciso di analizzare nelle successive sezioni solamente i meccanismi di autenticazione della rete
UMTS e 5G.\\
L'identificazione rappresenta un meccanismo molto vulnerabile ad attacchi di tipo \textit{Denial of Service} poichè, 
in alcuni casi, si riesce a consumare delle onerose operazioni computazionali anche con dispositivi che non sono abilitati,
quindi senza SIM.

\subsection{2G}
\subsection{3G}
\subsection{4G}
Il meccanismo presente nella rete UMTS è lo stesso usato nelle generazioni cellulari GSM, GPRS e EDGE. Inoltre, il 
suo funzionamento è molto simile a quello delle reti LTE (4G), pertanto si è deciso di presentarlo solamente una volta.\\
Un \acrshort{ms} che si vuole collegare alla rete deve procedere con la fase di autenticazione o identificazione anche detta \textit{Authentication and key agreement}
(AKA). In questa fase, viene interrogata la rispettiva HLR/AuC dove l'IMSI del dispositivo viene validato, se tutto procede correttamente
viene notificato il SGSN che inoltra al \acrshort{ms} l'avviso di autenticazione completata.
%Schema autenticazione umts

\subsection{5G}