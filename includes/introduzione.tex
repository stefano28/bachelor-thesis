\chapter{Introduzione}
Le reti cellulari rappresentano un punto nevralgico per le nostre comunicazioni.
Per questo, la loro sicurezza è fondamentale per garantire un normale funzionamento
di tutti i servizi a cui ormai ci siamo abituati.\\
La nuova tecnologia di quinta generazione è ormai vicina ad essere implementata su larga scala
per permettere lo sviuluppo del mondo dell'\gls{iot}. Questa nuova tecnologia stravolge numerosi
paradigmi strutturali che sono stati utilizzati fin'ora nelle generazioni precedenti, introducendo nuove sfide nell'ambito
della loro sicurezza.
\section{Struttura del documento}
Il documento è strutturato in modo da fornire al lettore le competenze e terminologie adeguate per comprendere tutti i dettagli delle 
vulnerabilità trattate.\\
L'elaborato inizia con una breve panoramica sulla rete cellulare, descrivendo genericamente la sua struttura e architettura.\\ 
Dato che le specifiche dell'architettura di una rete cellulare sono molto diverse a seconda della generazione, è stato 
necessario illustrare l'evoluzione delle varie tecnologie: da 1G a 5G. 
Per ogni generazione verranno illustratate prevalentemente le sue propietà architetturali oltre che le principali novità introdotte.\\
Successivamente, verrà introdotta la tipologia dell'attacco trattato, ossia il \gls{dos}, spiegando in cosa consiste
e come si applica alle reti cellulari. Inoltre, verranno illustrate le misurazioni necessarie per valutare l'efficienza di un attacco.\\
Tra le tante vulnerabilità che possono essere sfruttate per generare un \gls{dos}, questo documento si vuole soffermare sull'analisi 
dell'attacco all'autenticazione.
Per comprendere le vulnerabilità in questo ambito verranno analizzati nel dettaglio i sistemi di autenticazione per le varie generazioni cellulari.\\
Nel capitolo seguente verranno introdotti gli attacchi all'autenticazione delle reti 2G-4G, spiegando il loro funzionamento di base e citando vari studi che 
hanno permesso di quantificare la pericolosità di questi attacchi.\\
Infine, verranno analizzati gli attacchi \gls{dos} alle reti 5G, specificando quali miglioramenti o peggioramenti sono stati introdotti dalla sua nuova architettura.
\section{Scopo della tesi}
Questo elaborato vuole descrivere il funzionamento degli attacchi di tipo \gls{dos} alle reti cellulari, in particolare al meccanismo 
di autenticazione degli utenti.\\
Infine, si vuole scoprire se le vulnerabilità delle generazioni 2G-4G sono state risolte nel 5G.